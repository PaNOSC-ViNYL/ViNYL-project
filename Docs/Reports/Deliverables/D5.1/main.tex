\documentclass[11pt, a4paper]{article}
\usepackage[utf8]{inputenc}
\usepackage{graphicx}
\usepackage{setspace}
\usepackage{array}
\usepackage{fancyhdr}
\usepackage[]{biblatex}
\addbibresource{references.bib}
% Set date format
\usepackage[ddmmyyyy]{datetime}
% Set fonts
%\setmainfont{Muli}[
    %Path = fonts/ ,
    %Extension = .ttf ,
    %UprightFont = *-Regular ,
    %ItalicFont = *-Italic ,
    %BoldFont = *-Bold ,
    %BoldItalicFont = *-BoldItalic
%]
%\setsansfont{Muli-Black}[
    %Path = fonts/ ,
    %Extension = .ttf ,
    %ItalicFont = *Italic
%]
% Set page margin
\usepackage{geometry}
\geometry{a4paper,
            % total={175mm,257mm}, %space for content
            headheight=80pt,
            headsep=2pt,
            % top=3cm,
            bottom=2cm,
            left=2cm,
            right=1.75cm}
% Set the name of the table of contents
\usepackage{url}                                                  % for correct typesettings of URLs
\usepackage{hyperref}                                             % for sophisticated linking of urls, dois, pictures, tables, etc.
\hypersetup{
    unicode=true,                                                 % non-Latin characters in Acrobat’s bookmarks
    pdftoolbar=true,                                              % show Acrobat’s toolbar?
    pdfmenubar=true,                                              % show Acrobat’s menu?
    pdffitwindow=false,                                           % window fit to page when opened
    pdfstartview={FitH},                                          % fits the width of the page to the window
    pdfauthor={C. Fortmann-Grote, Juncheng E},                                           % author
    pdftitle={D5.1: Prototype simulation data formats as openPMD domain specific extensions including example datasets},   % title
    pdfsubject={PaNOSC WP5 (ViNYL) Deliverable D5.1},                             % subject of the document
    pdfcreator={pdflatex},                                         % creator of the document
    pdfnewwindow=true,                                            % links in new PDF window
    colorlinks=true,                                              % false: boxed links; true: colored links
    linkcolor=blue,                                                % color of internal links (change box color with linkbordercolor)
    citecolor=blue,                                                % color of links to bibliography
    filecolor=blue,                                               % color of file links
    urlcolor=blue                                                 % color of external links
}


\renewcommand{\contentsname}{Table of Contents}

\pagestyle{fancy}
\fancyhead{} % clear all header fields
\fancyfoot{} % clear all footer fields
\renewcommand{\headrulewidth}{0pt} % clear the header ruler line
\cfoot{\includegraphics[width=\textwidth]{footer.pdf}}

\begin{document}
% \thispagestyle{empty}
{
	\centering
    \begin{onehalfspace}
    \sffamily
	{\LARGE PaNOSC \par}
	{\LARGE Photon and Neutron Open Science Cloud \par}
	{\LARGE H2020-INFRAEOSC-04-2018 \par}
	{\LARGE Grant Agreement Number: 823852 \par}
	\end{onehalfspace}

	\vfill
	\includegraphics[width=\textwidth]{PaNOSClogo_web_RGB.pdf}\par
	\vfill

	{\large \textbf{Deliverable: 5.1 Prototype simulation data formats as openPMD domain specific extensions including example datasets}}
} % end of centering
\newpage

\chead{\includegraphics[width=\textwidth]{header.pdf}}
\fancyfoot{} % clear all footer fields
\fancyfoot[R]{\thepage}
{\sffamily\LARGE Project Deliverable Information Sheet \par}

\begin{center}
\begin{tabular}{ | m{5.2cm}| m{10.7cm} | }
\hline
Project Reference No. & 823852 \\
\hline
Project acronym: & PaNOSC \\
\hline
Project full name: & Photon and Neutron Open Science Cloud \\
\hline
H2020 Call: & INFRAEOSC-04-2018 \\
\hline
Project Website: & www.panosc.eu \\
\hline
Deliverable No: & D5.1 Prototype simulation data formats as openPMD domain specific extensions including example datasets \\
\hline
Deliverable Type: & Report \\
\hline
Dissemination Level: & Public \\
\hline
Contractual Delivery Date: & 31/11/2019 \\
\hline
Actual Delivery Date: & \today \\
\hline
EC project Officer: & Geert Vancraeynest \\
\hline
\end{tabular}
\end{center}

{\large \textbf{Document Control Sheet} \par}
\begin{center}
\begin{tabular}{ | m{5.2cm}| @{}c@{} | }
\hline
\textbf{Document}
&
\begin{tabular}{| m{10.7cm} |}
Title D5.1 Prototype simulation data formats as openPMD domain specific extensions including example datasets \\\hline
Version: 1 \\\hline
Available at: \\\hline
Files: 1 \\\hline
\end{tabular}
\tabularnewline\hline
\textbf{Authorship}
&
\begin{tabular}{| m{10.7cm} |}
Written by: Carsten Fortmann-Grote, Juncheng E \\\hline
Contributors: \\
Reviewed by: \\\hline
Approved:  \\\hline
\end{tabular}
\tabularnewline\hline
\end{tabular}
\end{center}

{\large \textbf{List of participants} \par}
\begin{center}
    \begin{tabular}{|m{3.0cm}|m{9.5cm}|m{3cm}|}
        \hline
        \textbf{Participant No.} & \textbf{Participant organisation name} & \textbf{Country} \\
        \hline
        1 & European Synchrotron Radiation Facility (ESRF) & France \\
        \hline
        2 & Institut Laue-Langevin (ILL) & France \\
        \hline
        3 & European XFEL (XFEL.EU) & Germany \\
        \hline
        4 & The European Spallation Source (ESS) & Sweden \\
        \hline
        5 & Extreme Light Infrastructure Delivery Consortium (ELI-DC) & Belgium \\
        \hline
        6 & Central European Research Infrastructure Consortium (CERIC-ERIC) & Italy \\
        \hline
        7 & EGI Foundation (EGI.eu) & The Netherlands \\
        \hline
    \end{tabular}
\end{center}

\addtocontents{toc}{\protect\thispagestyle{fancy}}
\tableofcontents
\newpage

\section{Introduction}
Standardized formatting and hierarchical organization of simulation data and associated metadata is paramount to support seamless
exchange of data between simulation software in simulation pipelines and to benefit from 3rd party data visualization
and analysis software that support these formats. By applying the openPMD metadata standard for particle and mesh data \cite{Huebl2015} to all simulated data in
WP5, our workflows and results become accessible, inter--operable, and reusable, i.e. in line with the core concepts of
FAIR Data Principles \cite{Wilkinson2016}.

This report details the metadata standard extensions for simulation data developed in Work Package 5 (WP5) of the Photon
and Neutron Open Science Cloud (PaNOSC). The openPMD standard in its current version 1.1 is the basis on which we have
developed domain specific extensions for the simulation stages of start--to--end photon and neutron experiment
simulations:
\begin{enumerate}
  \item Coherent wavefront propagation (simulation of coherent lightsources and beam transport)
  \item Photon raytracing (simulation of x--ray optical beamline components, incoherent sources)
  \item Neutron raytracing (simulation of neutron beamlines)
  \item Molecular dynamics simulations (target simulations, radiation--matter interaction)
\end{enumerate}

All extensions are deposited under the WP5 github project at \url{https://github.com/PaNOSC-ViNYL/openPMD-standard} in
the branch ``upcoming-2.0'' in fulfillment of the project deliverable D5.1 of PaNOSC. To expose our standard extensions
to the wider community, we will propose to merge them into the base standard repository
\url{https://github.com/openPMD/openPMD-standard} after the Deliverable has been accepted.

\section{The openPMD base standard}
OpenPMD stands for open particle and mesh data and was initially developed as a metadata and data hierarchy standard for
particle--in--cell (PIC) simulations of high--power laser--matter interaction at the Helmholtz--Zentrum Dresden--Rossendorf.
Due to its flexibility (allowing for adaptation and adjustments for other applications than PIC), independence from the
actual file format (supporting e.g. hdf5 \cite{HDFGroup1997}, netCDF \cite{Rew1989}, adios \cite{Liu2014} and json
\cite{json2019} and
more), and being maintained by an active open source community, the openPMD standard is widely adopted in numerous
simulation codes, visualization codes, and simulation workflow platforms.

The base standard structures simulation data by the following characteristics:
\begin{enumerate}
  \item The Series: The root object of the data hierarchy.
  \item Iteration: Simulation time step (the time stamp of a given simulation snapshot)
  \item Particle  or mesh
  \item Type of particle (e.g. electron, proton) or mesh (e.g. electric field, magnetic field).
  \item Records (Physical variables):
    \begin{itemize}
      \item For particles: Position and momentum (velocity) vector components
      \item For meshes: Field values (components for vector fields)
    \end{itemize}
\end{enumerate}
In addition, the standard defines a number of mandatory and optional metadata information to define and specify the
physical meaning of the data. E.g. for particle data, each record must provide information about the physical unit in
which its data is written. A mesh must provide the grid spacing, dimensionality, extension, and geometry and, where
appropriate, the corresponding units. An iteration must provide which physical time it corresponds to. Further metadata
about the code that produced the data, the data author, the time and date of the simulation, the version of the standard
that this data was formatted against, are provided as attributes of the Series.

\section{OpenPMD extensions}

\subsection{Wavefront data}
Coherent wavefront propagation plays an import role in the simulation of coherent light sources, such as free--electron
lasers. Applications include studying the intensity and phase distributions at the experimental interaction point and
effects of misalignments, pointing errors, fluctuations in the temporal, spatial, and spectral pulse structure on the
latter.

During a wavefront simulation, the horizontal and vertical polarization components of the complex electric field is calculated at a given position in the path of the light
pulse in three dimensions: horizontal and vertical position in a plane perpendicular to the beam axis and time. From
this data, all observables such as intensity, phase, polarization, divergence, pointing, energy spectrum and other, can
be calculated. The information is complete in the sense, that the propagation can be continued further down the
beamline from a simulation that stopped at an upstream position. The temporal sampling resolves the slowly varying pulse
envelope but not every single oscillation of the the laser fundamental frequency.

The electric field components as defined above are the required data records for the wavefront domain extension.
Besides, the standard requires information about the radius of curvature in horizontal and in vertical direction, its
standard deviation, as well as the standard photon energy to be given as attributes of the Series. Furthermore, it must
be specified whether the data represents the pulse in the time or in the frequency domain, and in the cartesian, or in
the reciprocal space.

This domain extension can be found at \url{https://github.com/PaNOSC-ViNYL/openPMD-standard/blob/upcoming-2.0.0/EXT_WAVEFRONT.md}.
An example dataset is published on zenodo at \ldots \marginpar{TODO: Carsten}.

\subsection{Raytracing data}
\subsubsection{Neutrons}
\marginpar{TODO: Mads }
\subsubsection{Photons}
\marginpar{TODO: Aljosa }
\subsection{Molecular Dynamics data}
\marginpar{TODO: Juncheng }
\fancypagestyle{plain}{}
\printbibliography
\end{document}
