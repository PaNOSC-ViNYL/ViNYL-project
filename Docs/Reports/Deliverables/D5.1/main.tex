\documentclass[11pt, a4paper]{report}
\usepackage[utf8]{inputenc}
\usepackage{graphicx}
\usepackage{natbib}
\usepackage{setspace}
\usepackage{array}
\usepackage{fancyhdr}
% Set date format
\usepackage[ddmmyyyy]{datetime}
% Set fonts
\usepackage{fontspec}
\setmainfont{Muli}[
    Path = fonts/ ,
    Extension = .ttf ,
    UprightFont = *-Regular ,
    ItalicFont = *-Italic ,
    BoldFont = *-Bold ,
    BoldItalicFont = *-BoldItalic
]
\setsansfont{Muli-Black}[
    Path = fonts/ ,
    Extension = .ttf ,
    ItalicFont = *Italic
]
% Set page margin
\usepackage{geometry}
\geometry{a4paper,
            % total={175mm,257mm}, %space for content
            headheight=80pt,
            headsep=2pt,
            % top=3cm,
            bottom=2cm,
            left=2cm,
            right=1.75cm}
% Set the name of the table of contents
\renewcommand{\contentsname}{Table of Contents}

\pagestyle{fancy}
\fancyhead{} % clear all header fields
\fancyfoot{} % clear all footer fields
\renewcommand{\headrulewidth}{0pt} % clear the header ruler line 
\cfoot{\includegraphics[width=\textwidth]{footer.pdf}}

\begin{document}
% \thispagestyle{empty}
{
	\centering
    \begin{onehalfspace}
    \sffamily
	{\LARGE PaNOSC \par}
	{\LARGE Photon and Neutron Open Science Cloud \par}
	{\LARGE H2020-INFRAEOSC-04-2018 \par}
	{\LARGE Grant Agreement Number: 823852 \par}
	\end{onehalfspace}
	
	\vfill
	\includegraphics[width=\textwidth]{PaNOSClogo_web_RGB.pdf}\par
	\vfill

	{\large \textbf{Deliverable: 5.1 Prototype simulation data formats as openPMD domain specific extensions including example datasets}}
} % end of centering
\newpage

\chead{\includegraphics[width=\textwidth]{header.pdf}}
\fancyfoot{} % clear all footer fields
\fancyfoot[R]{\thepage}
{\sffamily\LARGE Project Deliverable Information Sheet \par}

\begin{center}
\begin{tabular}{ | m{5.2cm}| m{10.7cm} | }
\hline
Project Reference No. & 823852 \\ 
\hline
Project acronym: & PaNOSC \\ 
\hline
Project full name: & Photon and Neutron Open Science Cloud \\ 
\hline
H2020 Call: & INFRAEOSC-04-2018 \\ 
\hline
Project Website: & www.panosc.eu \\ 
\hline
Deliverable No: & D5.1 Prototype simulation data formats as openPMD domain specific extensions including example datasets \\ 
\hline
Deliverable Type: & Report \\ 
\hline
Dissemination Level: & Public \\ 
\hline
Contractual Delivery Date: & 31/11/2019 \\ 
\hline
Actual Delivery Date: & \today \\
\hline
EC project Officer: & Geert Vancraeynest \\ 
\hline
\end{tabular}
\end{center}

{\large \textbf{Document Control Sheet} \par}
\begin{center}
\begin{tabular}{ | m{5.2cm}| @{}c@{} | }
\hline
\textbf{Document}
& 
\begin{tabular}{| m{10.7cm} |}
Title D5.1 Prototype simulation data formats as openPMD domain specific extensions including example datasets \\\hline
Version: 1 \\\hline
Available at: \\\hline
Files: 1 \\\hline
\end{tabular}
\tabularnewline\hline
\textbf{Authorship} 
& 
\begin{tabular}{| m{10.7cm} |}
Written by: Carsten Fortmann-Grote, Juncheng E \\\hline
Contributors: \\
Reviewed by: \\\hline
Approved:  \\\hline
\end{tabular}
\tabularnewline\hline
\end{tabular}
\end{center}

{\large \textbf{List of participants} \par}
\begin{center}
    \begin{tabular}{|m{3.0cm}|m{9.5cm}|m{3cm}|}
        \hline
        \textbf{Participant No.} & \textbf{Participant organisation name} & \textbf{Country} \\
        \hline
        1 & European Synchrotron Radiation Facility (ESRF) & France \\
        \hline
        2 & Institut Laue-Langevin (ILL) & France \\
        \hline
        3 & European XFEL (XFEL.EU) & Germany \\
        \hline
        4 & The European Spallation Source (ESS) & Sweden \\
        \hline
        5 & Extreme Light Infrastructure Delivery Consortium (ELI-DC) & Belgium \\
        \hline
        6 & Central European Research Infrastructure Consortium (CERIC-ERIC) & Italy \\
        \hline
        7 & EGI Foundation (EGI.eu) & The Netherlands \\
        \hline
    \end{tabular}
\end{center}

\addtocontents{toc}{\protect\thispagestyle{fancy}}
\tableofcontents
\newpage

\addcontentsline{toc}{section}{Introduction}
\section*{Introduction}
There is a theory which states that if ever anyone discovers exactly what the Universe is for and why it is here, it will instantly disappear and be replaced by something even more bizarre and inexplicable.
There is another theory which states that this has already happened.

\begin{figure}[h!]
\centering
\includegraphics[scale=1.7]{universe}
\caption{The Universe}
\label{fig:universe}
\end{figure}

\addcontentsline{toc}{section}{Conclusion}
\section*{Conclusion}
``I always thought something was fundamentally wrong with the universe'' \citep{adams1995hitchhiker}

\fancypagestyle{plain}{}
\bibliographystyle{plain}
\bibliography{references}
\end{document}
