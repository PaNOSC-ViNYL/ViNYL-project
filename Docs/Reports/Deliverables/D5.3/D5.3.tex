\documentclass[10pt]{scrartcl}
\usepackage[english]{babel}
\usepackage{multirow}
\usepackage[default]{opensans}
\usepackage{sfmath} % sans font also for math
\usepackage[binary-units = true]{siunitx}
\usepackage{graphicx}
% defining the paper layout that no text overlaps with the header
\usepackage[
  top=35mm,
  headheight=25mm,
  headsep=3mm,
  bottom=30mm,
  left=0.7in,
  right=0.7in
]{geometry}

\usepackage[verbose]{placeins}
\usepackage{subcaption}
\usepackage{latexsym}
\usepackage[centertags]{amsmath}
\usepackage{amssymb}
\usepackage[]{glossaries}

\graphicspath{{figures/}}
% custom header and footpage
\usepackage{scrlayer-scrpage}
\pagestyle{scrheadings} % you have to set the custom layout
% Head
\chead{}
\ihead{\includegraphics[width=0.5in]{figures/PaNOSClogo_web_RGB_512x512.jpg}}
\ohead{}%
% Foot
\ofoot{} % left foot
\cfoot{%
  \normalfont{%
    \begin{tiny}%
      \begin{centering}
        This project has received funding from the
        European Union’s Horizon 2020 research and innovation programme
        under grant agreement No 823852.
        \begin{minipage}[c]{0.3in}
          \includegraphics[width=\textwidth]{figures/EU.png}
        \end{minipage}
      \end{centering}%
    \end{tiny}%
  }%
}%
%
\usepackage{booktabs}
%
%%%%%%%%%%%%%%%%%%%%%%%%%%%%%%%%%%%%%%%%%%%%%%%
%   BIBLIOGRAPHY SETTINGS
\usepackage[bibstyle=nature,sorting=none,maxnames=1000,eprint=false,
defernumbers=true, backend=biber]{biblatex}
\usepackage{chemformula}
\usepackage{hyperref}


%\renewcommand*\finalnamedelim{, and\addspace}
%\DeclareNameAlias{sortname}{last-first}
%\renewcommand{\newunitpunct}{, }

%\AtEveryBibitem{%
  %\clearfield{day}%
  %\clearfield{month}%
  %\clearfield{endday}%
  %\clearfield{endmonth}%
  %\clearfield{issn}%
  %\clearfield{issue}%
%}
%%convert titles to hyperlinks using doi
%\ExecuteBibliographyOptions{doi=true} \newbibmacro{string+doi}[1]{%
  %\iffieldundef{doi}{#1}{\href{http://dx.doi.org/\thefield{doi}}{#1}}}
  %\DeclareFieldFormat*{title}{\usebibmacro{string+doi}{\mkbibemph{#1}}}

%\addbibresource{urls.bib}
%\addbibresource{footnotes.bib}
\addbibresource{references.bib}

%%%%%%%%%%%%%%%%%%%%%%%%%%%%%%%%%%%%%%%%%%%%%%%
% GLOSSARY SETTINGS
%\setacronymstyle{long-short}
%\input{glossary}
%\makeglossaries
%%%%%%%%%%%%%%%%%%%%%%%%%%%%%%%%%%%%%%%%%%%%%%%

% Zeilenabstand
\renewcommand{\baselinestretch}{1.2}


% sophisticated linking of references in the pdf and setting some options
\usepackage{url}                                                  % for correct typesettings of URLs
\usepackage{hyperref}                                             % for sophisticated linking of urls, dois, pictures, tables, etc.
\hypersetup{
    unicode=true,                                                 % non-Latin characters in Acrobat’s bookmarks
    pdftoolbar=true,                                              % show Acrobat’s toolbar?
    pdfmenubar=true,                                              % show Acrobat’s menu?
    pdffitwindow=false,                                           % window fit to page when opened
    pdfstartview={FitH},                                          % fits the width of the page to the window
    pdfauthor={C. Fortmann-Grote},                                           % author
    pdftitle={D5.3: Repository of documented jupyter notebooks and Oasys canvase },   % title
    pdfsubject={PaNOSC WP5 (ViNYL) Deliverable D5.3},                             % subject of the document
    pdfcreator={pdflatex},                                         % creator of the document
    pdfkeywords={PaNOSC, ViNYL, simulations, jupyter, simex, oasys, McStas,
    service},                                         % list of keywords
    pdfnewwindow=true,                                            % links in new PDF window
    colorlinks=true,                                              % false: boxed links; true: colored links
    linkcolor=blue,                                                % color of internal links (change box color with linkbordercolor)
    citecolor=blue,                                                % color of links to bibliography
    filecolor=blue,                                               % color of file links
    urlcolor=blue                                                 % color of external links
}

\begin{document}
\makeatletter
\begin{titlepage}
\thispagestyle{scrheadings}
\ohead{}
\ihead{}
\chead{}
\ifoot{}
\ofoot{}
\noindent%
\includegraphics[width=0.4\textwidth]{figures/PaNOSClogo_web_RGB_512x512.jpg}\\
\Huge{%
\renewcommand{\baselinestretch}{2.0}%
  \textbf{%
    Deliverable D5.3: Repository of documented jupyter notebooks and Oasys canvases\\
  }%
}%
\\
{%
\Large{%
  Mads Bertelsen,
  Stella d'Ambrumenil,
  Juncheng E,
  Aljosa Hafner,
  Gergely Norbert Nagy,
  Shervin Nourbakhsh,
  Mousumi Upadhyay Kahaly,
  Carsten Fortmann-Grote
  \bigskip\\
  \bigskip\\
  \textbf{\today}%
}}%
\end{titlepage}
\makeatother

%\tableofcontents
\section{Introduction}
% Proposed introduction by Mads
This document describes the activities of PaNOSC work package 5 (WP5) that fall under the D5.3 deliverable on documented simulation data services for beamline optics. WP5 has developed libpyvinyl as a basis interface used for simulation of X-ray and Neutron scattering, this library ensures a common interface for simulation that choose to build upon this foundation. That common interface have enabled us to create a consistent instrument database that can contain both X-ray and Neutron instruments. Instrument can be easily retrieved from an accompanying Python API, and expert users can update the instrument description directly though github that host the actual database. By hosting the instrument database on GitHub we both utilise the standard tool for the field and get the option of running quality checks on proposed updates of instruments using Continuous Integration.

In addition to the common instrument database each package have a dedicated github repository with Jupyter Notebooks that demonstrate both what can be achieved with the package and how to do so. The content of each are described in turn.
\\
TBD: Carsten, All

\section{Oasys workspace repository}
\label{sec:oasys}
TBD: Aljosa

\section{McStasScript notebooks}
\label{sec:mcstas}
The McStasScript notebook repository along with the code repository are shown below, along with a link to the online documentation.
\begin{center}
\begin{tabular}{l l}
McStasScript code repo  & \url{https://github.com/PaNOSC-ViNYL/McStasScript} \\ 
McStasScript notebooks & \url{https://github.com/PaNOSC-ViNYL/McStasScript-notebooks} \\  
McStasScript documentation & \url{https://mads-bertelsen.github.io}
\end{tabular}
\end{center}
The McStasScript notebook repository currently contains a full 11 notebook tutorial on McStas using the McStasScript Python interface developed under WP5. This tutorial is also available through the online McStasScript documentation. The repository also contains an example on using the widget interface for McStasScript, and an example on using a cryostat construction tool included in McStasScript.

\section{SimEx notebooks}
\label{sec:simex}
TBD: Jun, Carsten

simex - shadow/oasys


\end{document}

