\documentclass[10pt]{scrartcl}
\usepackage[english]{babel}
\usepackage{multirow}
\usepackage[default]{opensans}
\usepackage{sfmath} % sans font also for math
\usepackage[binary-units = true]{siunitx}
\usepackage{graphicx}
% defining the paper layout that no text overlaps with the header
\usepackage[
  top=35mm,
  headheight=25mm,
  headsep=3mm,
  bottom=30mm,
  left=0.7in,
  right=0.7in
]{geometry}

\usepackage[verbose]{placeins}
\usepackage{subcaption}
\usepackage{latexsym}
\usepackage[centertags]{amsmath}
\usepackage{amssymb}
\usepackage[]{glossaries}

\graphicspath{{figures/}}
% custom header and footpage
\usepackage{scrlayer-scrpage}
\pagestyle{scrheadings} % you have to set the custom layout
% Head
\chead{}
\ihead{\includegraphics[width=0.5in]{figures/PaNOSClogo_web_RGB_512x512.jpg}}
\ohead{}%
% Foot
\ofoot{} % left foot
\cfoot{%
  \normalfont{%
    \begin{tiny}%
      \begin{centering}
        This project has received funding from the
        European Union’s Horizon 2020 research and innovation programme
        under grant agreement No 823852.
        \begin{minipage}[c]{0.3in}
          \includegraphics[width=\textwidth]{figures/EU.png}
        \end{minipage}
      \end{centering}%
    \end{tiny}%
  }%
}%
%
\usepackage{booktabs}
%
%%%%%%%%%%%%%%%%%%%%%%%%%%%%%%%%%%%%%%%%%%%%%%%
%   BIBLIOGRAPHY SETTINGS
\usepackage[bibstyle=nature,sorting=none,maxnames=1000,eprint=false,
defernumbers=true, backend=biber]{biblatex}
\usepackage{chemformula}
\usepackage{hyperref}


%\renewcommand*\finalnamedelim{, and\addspace}
%\DeclareNameAlias{sortname}{last-first}
%\renewcommand{\newunitpunct}{, }

%\AtEveryBibitem{%
  %\clearfield{day}%
  %\clearfield{month}%
  %\clearfield{endday}%
  %\clearfield{endmonth}%
  %\clearfield{issn}%
  %\clearfield{issue}%
%}
%%convert titles to hyperlinks using doi
%\ExecuteBibliographyOptions{doi=true} \newbibmacro{string+doi}[1]{%
  %\iffieldundef{doi}{#1}{\href{http://dx.doi.org/\thefield{doi}}{#1}}}
  %\DeclareFieldFormat*{title}{\usebibmacro{string+doi}{\mkbibemph{#1}}}

%\addbibresource{urls.bib}
%\addbibresource{footnotes.bib}
\addbibresource{references.bib}

%%%%%%%%%%%%%%%%%%%%%%%%%%%%%%%%%%%%%%%%%%%%%%%
% GLOSSARY SETTINGS
%\setacronymstyle{long-short}
%\input{glossary}
%\makeglossaries
%%%%%%%%%%%%%%%%%%%%%%%%%%%%%%%%%%%%%%%%%%%%%%%

% Zeilenabstand
\renewcommand{\baselinestretch}{1.2}


% sophisticated linking of references in the pdf and setting some options
\usepackage{url}                                                  % for correct typesettings of URLs
\usepackage{hyperref}                                             % for sophisticated linking of urls, dois, pictures, tables, etc.
\hypersetup{
    unicode=true,                                                 % non-Latin characters in Acrobat’s bookmarks
    pdftoolbar=true,                                              % show Acrobat’s toolbar?
    pdfmenubar=true,                                              % show Acrobat’s menu?
    pdffitwindow=false,                                           % window fit to page when opened
    pdfstartview={FitH},                                          % fits the width of the page to the window
    pdfauthor={C. Fortmann-Grote},                                           % author
    pdftitle={MS5.2: Demonstration of simulation services},   % title
    pdfsubject={PaNOSC WP5 (ViNYL) Milestone MS5.2},                             % subject of the document
    pdfcreator={pdflatex},                                         % creator of the document
    pdfkeywords={PaNOSC, ViNYL, simulations, jupyter, simex, oasys, McStas,
    service},                                         % list of keywords
    pdfnewwindow=true,                                            % links in new PDF window
    colorlinks=true,                                              % false: boxed links; true: colored links
    linkcolor=blue,                                                % color of internal links (change box color with linkbordercolor)
    citecolor=blue,                                                % color of links to bibliography
    filecolor=blue,                                               % color of file links
    urlcolor=blue                                                 % color of external links
}

\begin{document}
\makeatletter
\begin{titlepage}
\thispagestyle{scrheadings}
\ohead{}
\ihead{}
\chead{}
\ifoot{}
\ofoot{}
\noindent%
\includegraphics[width=0.4\textwidth]{figures/PaNOSClogo_web_RGB_512x512.jpg}\\
\Huge{%
\renewcommand{\baselinestretch}{2.0}%
  \textbf{%
    Milestone MS5.2: Demonstration of simulation services\\
  }%
}%
\\
{%
\Large{%
  Carsten Fortmann-Grote,
  Mads Bertelsen,
  Thomas Holm Rod,\\
  Manuel Sanchez-Rio,
  Tom Schoonjans
  \bigskip\\
  \bigskip\\
  \textbf{\today}%
}}%
\end{titlepage}
\makeatother

%\tableofcontents
\section{Introduction}
% This is the milestone report for the task, demonstration of simulation services
% We have the following packages
% 

% This is a report for milestone and work package 5
% Responsibility of the work package: harmonize API's to simulation
% 

This report outlines the documented simulation APIs released under PaNOSC work package 5 and constitutes our second milestone. The work package is internally named the Virtual Neutron and X--Ray Laboratory ``ViNYL'', and is responsible for exposing simulation packages in a cloud environment with a unified user interface. Here we demonstrate the underlying APIs that will provide users access to the simulation packages through a documented python interface.

The major simulation packages exposed in ViNYL are simex, McStas and Oasys. Simex is a python based tool for simulation of X-ray beamlines, and connects to a large number of simulation backengines. In some ways, simex serves as the template for how ViNYL can later harmonize access to more simulation packages.

McStas is a neutron instrumentation simulation package that uses a meta-language built on C as a user interface, which is more difficult to expose to a cloud based solution. Work package 5 is providing a python API for McStas called McStasScript which is more suitable for exposing through a cloud service such as a Jupyter Hub.

The last major project in ViNYL is Oasys which provides a graphical user interface to simulation of X-ray beamlines. This project will be made available through remote desktop software due to the need for the graphical user interface.

With the APIs ready, ViNYL is now ready to harmonize these into the collected pyvinyl package that will provide an interface to simulation of both X-ray and neutron beamlines.

%A major objective of work package 5 is to harmonize the APIs into a single interface,

\section{SiMEX}
% Introduction to the package, why is it in work package 5

\subsection{Capabilities}
% What kind of simulations can be performed with this user interface

\subsection{Documentation}
% Overview of the available documentation, with links if appropriate

\subsection{Code availablity}
% Where is the code developed (github repo), how is it installed and where is it available?

\section{oasys}
% Introduction to the package, why is it in work package 5

\subsection{Capabilities}
% What kind of simulations can be performed with this user interface

\subsection{Documentation}
% Overview of the available documentation, with links if appropriate

\subsection{Code availablity}
% Where is the code developed (github repo), how is it installed and where is it available?

\section{McStasScript}
McStasScript is a python API to the McStas package which is capable of simulating neutron scattering instrumentation. Using McStasScript, McStas simulations can be built and executed directly from a python interface. The resulting data from the simulations are made available as python data structures. These tasks are appropriate for Jupyter Notebooks which are more suitable to being exposed as a cloud service than the traditional McStas workflow.

\subsection{Capabilities}
McStasScript is an API to McStas, and covers nearly the entire set of McStas features. McStas is widely used for design of neutron scattering instrumentation, usually used for simulation from the neutron moderator to detector including sample physics. The package includes the tools necessary for simulating optics, samples of different types, sample environment and detectors.

\subsection{Documentation}
Documentation for McStasScript is provided in several forms that suits different kinds of users and developers. Some understanding of the underlying McStas package is necessary, the basics of which are available through the Jupyter Notebook tutorials.

\begin{itemize}
  \item \href{https://github.com/PaNOSC-ViNYL/McStasScript/blob/master/McStasScript_developer_documentation.pdf}{Classic manual} in pdf format outlining all userfacing classes and methods
  \item \href{https://github.com/PaNOSC-ViNYL/McStasScript/blob/master/McStasScript_documentation.pdf}{Developer documentation} in pdf format explaining development principles and class diagram
  \item \href{https://github.com/PaNOSC-ViNYL/McStasScript/tree/master/tutorial}{Set of 11 Jupyter notebooks} that demonstrates and teaches McStas and McStasScript
  \item Documentation strings in python classes and methods that provide help during use
\end{itemize}

\subsection{Code availablity}
The McStasScript API is a pure python package, and is distributed with pypi making installation easy. McStas is to be installed separately, and McStasScript needs to be configured to find the underlying McStas installation. McStas and McStasScript are also available together in a single docker container where configuration is already performed.

\begin{tabular}{|l|l|l|l|}
  \hline
  \textbf{Description} & \textbf{Link} \\
  \hline
  McStas GitHub repository & \url{https://github.com/McStasMcXtrace/McCode} \\
  McStasScript GitHub repository & \url{https://github.com/PaNOSC-ViNYL/McStasScript} \\
  McStasScript pypi page & \url{https://pypi.org/project/McStasScript/} \\
  McStas + McStasScript docker image & \url{https://hub.docker.com/r/mccode/mcstas-2.6.1-mcstasscript} \\  
  \hline
\end{tabular}


\end{document}

