\documentclass[10pt]{scrartcl}
\usepackage[english]{babel}
\usepackage{multirow}
\usepackage[default]{opensans}
\usepackage{sfmath} % sans font also for math
\usepackage[binary-units = true]{siunitx}
\usepackage{graphicx}
% defining the paper layout that no text overlaps with the header
\usepackage[
  top=35mm,
  headheight=25mm,
  headsep=3mm,
  bottom=30mm,
  left=0.7in,
  right=0.7in
]{geometry}

\usepackage[verbose]{placeins}
\usepackage{subcaption}
\usepackage{latexsym}
\usepackage[centertags]{amsmath}
\usepackage{amssymb}
\usepackage[]{glossaries}

\graphicspath{{figures/}}
% custom header and footpage
\usepackage{scrlayer-scrpage}
\pagestyle{scrheadings} % you have to set the custom layout
% Head
\chead{}
\ihead{\includegraphics[width=0.5in]{figures/PaNOSClogo_web_RGB_512x512.jpg}}
\ohead{}%
% Foot
\ofoot{} % left foot
\cfoot{%
  \normalfont{%
    \begin{tiny}%
      \begin{centering}
        This project has received funding from the
        European Union’s Horizon 2020 research and innovation programme
        under grant agreement No 823852.
        \begin{minipage}[c]{0.3in}
          \includegraphics[width=\textwidth]{figures/EU.png}
        \end{minipage}
      \end{centering}%
    \end{tiny}%
  }%
}%
%
\usepackage{booktabs}
%
%%%%%%%%%%%%%%%%%%%%%%%%%%%%%%%%%%%%%%%%%%%%%%%
%   BIBLIOGRAPHY SETTINGS
\usepackage[bibstyle=nature,sorting=none,maxnames=1000,eprint=false,
defernumbers=true, backend=biber]{biblatex}
\usepackage{chemformula}
\usepackage{hyperref}


%\renewcommand*\finalnamedelim{, and\addspace}
%\DeclareNameAlias{sortname}{last-first}
%\renewcommand{\newunitpunct}{, }

%\AtEveryBibitem{%
  %\clearfield{day}%
  %\clearfield{month}%
  %\clearfield{endday}%
  %\clearfield{endmonth}%
  %\clearfield{issn}%
  %\clearfield{issue}%
%}
%%convert titles to hyperlinks using doi
%\ExecuteBibliographyOptions{doi=true} \newbibmacro{string+doi}[1]{%
  %\iffieldundef{doi}{#1}{\href{http://dx.doi.org/\thefield{doi}}{#1}}}
  %\DeclareFieldFormat*{title}{\usebibmacro{string+doi}{\mkbibemph{#1}}}

%\addbibresource{urls.bib}
%\addbibresource{footnotes.bib}
\addbibresource{references.bib}

%%%%%%%%%%%%%%%%%%%%%%%%%%%%%%%%%%%%%%%%%%%%%%%
% GLOSSARY SETTINGS
%\setacronymstyle{long-short}
%\input{glossary}
%\makeglossaries
%%%%%%%%%%%%%%%%%%%%%%%%%%%%%%%%%%%%%%%%%%%%%%%

% Zeilenabstand
\renewcommand{\baselinestretch}{1.2}


% sophisticated linking of references in the pdf and setting some options
\usepackage{url}                                                  % for correct typesettings of URLs
\usepackage{hyperref}                                             % for sophisticated linking of urls, dois, pictures, tables, etc.
\hypersetup{
    unicode=true,                                                 % non-Latin characters in Acrobat’s bookmarks
    pdftoolbar=true,                                              % show Acrobat’s toolbar?
    pdfmenubar=true,                                              % show Acrobat’s menu?
    pdffitwindow=false,                                           % window fit to page when opened
    pdfstartview={FitH},                                          % fits the width of the page to the window
    pdfauthor={C. Fortmann-Grote},                                           % author
    pdftitle={MS5.3: VINYL Software Release },   % title
    pdfsubject={PaNOSC WP5 (ViNYL) Milestone MS5.3},                             % subject of the document
    pdfcreator={pdflatex},                                         % creator of the document
    pdfkeywords={PaNOSC, ViNYL, simulations, jupyter, simex, oasys, McStas,
    service},                                         % list of keywords
    pdfnewwindow=true,                                            % links in new PDF window
    colorlinks=true,                                              % false: boxed links; true: colored links
    linkcolor=blue,                                                % color of internal links (change box color with linkbordercolor)
    citecolor=blue,                                                % color of links to bibliography
    filecolor=blue,                                               % color of file links
    urlcolor=blue                                                 % color of external links
}

\begin{document}
\makeatletter
\begin{titlepage}
\thispagestyle{scrheadings}
\ohead{}
\ihead{}
\chead{}
\ifoot{}
\ofoot{}
\noindent%
\includegraphics[width=0.4\textwidth]{figures/PaNOSClogo_web_RGB_512x512.jpg}\\
\Huge{%
\renewcommand{\baselinestretch}{2.0}%
  \textbf{%
    Milestone MS5.3: VINYL Software Release\\
  }%
}%
\\
{%
\Large{%
  Mads Bertelsen,
  Stella d'Ambrumenil,
  Juncheng E,
  Aljosa Hafner,
  Gergely Norbert Nagy,
  Shervin Nourbakhsh,
  Mousumi Upadhyay Kahaly,
  Carsten Fortmann-Grote
  \bigskip\\
  \bigskip\\
  \textbf{\today}%
}}%
\end{titlepage}
\makeatother

%\tableofcontents
\section{Introduction}
%Motivation, general layout, relation to other projects When writing a
simulation package, a significant effort is required to set up an appropriate
structure and to streamline flow of information from the user to the individual
parts of the simulation. Traditionally each simulation package will solve these
tasks in a different manner, resulting in some unique user interface and
structure. One important accomplishment of PaNOSC workpackage 5 is to provide a
base package for new simulation packages that provides a suitable foundation
with solutions to all these basic problems, yet enough flexibility to
incorporate truly new functionality. This base package is called
\textit{libpyvinyl}. Simulation packages built on \textit{libpyvinyl} will be
able to skip basics tasks on parameters and software structure and go straight
to implementing interesting scattering physics. Furthermore it will adhere to a
standard scheme, making developers from other packages built on the same
foundation more capable of helping each other. The most crucial benefit is
however that users will find the interfaces of packages built on
\textit{libpyvinyl} similar and thus what they learn using one package will be
transferable to the others. These are the benefits of standardisation and
harmonisation. The SimEx and McStasScript packages that simulates X-ray and
Neutron instrumentation respectively are already using \textit{libpyvinyl}.

\section{Libpyvinyl and its components}
\label{sec:libpyvinyl}
Since \textit{libpyvinyl} need to serve as a flexible foundation, it is
important that the layout is general. The overall structure is breifly outlined
here before the major components are each described in their separate sections.
The basic unit in a simulation is referred to as a calculator as it performs
some calculation. The developer can choose what that calculation depend on, and
what this calculation results in, meaning there is flexibility both on input and
output. The input to a calculator could be output from another, but also
commonly parameters, for which there is a rich base class. A full simulation can
be thought of as a sequence of calculators, and conceptually this is referred to
as an instrument object, which also includes some additional capabilities.

\subsection{Calculators}
\label{sec:calculators}
The calculator is a very general class that can take input and can provide
output. All code that affect the physics of the simulation is contained within
such calculators, it could for example be a calculator for a piece of optics or
a sample. The input of a calculator can be data in the form of one or more data
objects \ref{sec:data} and a input for a number of parameters
\ref{sec:parameters}. The output is also organised as data objects. When a new
calculator is developed, the dependence on the external data and parameters is
specified.

\subsection{Data}
\label{sec:data}
The \textit{libpyvinyl} package provides baseclasses for both individual data
objects and collections of data objects. The data objects are very general and
the code mainly facilitates the capabilities needed to transfer data from one
calculator to another.

\subsection{Parameters}
\label{sec:parameters}
The individual parameters are usually defined by a calculator class using the
\textit{libpyvinyl} base parameter class. This class includes the ability to
contain arrays, unit and limits on the allowed value. This rich parameter is a
major part of the safety net that prevents a user in providing garbage input to
a calculator that lie outside of its assumptions, for example a negative
wavelength.

\subsection{Instruments}
\label{sec:instruments}
Instruments are conceptually a series of calculators but can also contain som
master parameters. One could imagine a source calculator, some optics
calculators, a sample calculator and then a detector calculator along with a
parameter describing the simulated energy. As a master parameter, information
about the simulated energy can be provided to multiple calculators that would
need that information to avoid the user having to specify redundant information.
The chain of calculators can then be executed from the instrument level,
providing the requested data.

\section{Installation}
\label{sec:installation}

\section{Instructions for developers}
\label{sec:dev}

\section{Instructions for users}
\label{sec:users}







\end{document}

